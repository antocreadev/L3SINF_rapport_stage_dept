\subsection{Participation à l'entreprise}

Dept à de nombreux clients dans l'ecommerce et cherche à concevoir des solutions pour automatiser, éviter de la répétition de taches et améliorer la cohésion entre les différents professionnelles dans le processus de développement et de création
\\ \\
L’équipe qui s’occupe du développement de cet solution est composé de trois personnes : 
\begin{itemize}
    \item M. Derek Brady est le directeur créatif et associé chez DEPT, il s'occupe sur la vision à long terme de l'entreprise, il supervise et à dirige la création et le développement des aspects visuels et créatifs des projets.
    \item M. Marc Raffalli est développeur front-end senior, il s'occupe de l'aspects techniques des projets, il s'est occupé de guider et de m'encadrer tout au long du stage
    \item Enfin, Moi-même, stagiaire ayant pour but de créer la solution pour optimier le processus de création avec l'UI et l'UX dynamique.
\end{itemize}

\section{Cahier des charges}
Le but de ce stage est de créer une solution pour automatiser pour la création d'UI et UX dynamique avec React et Next.js pour les futurs clients de l'entreprise, en utilisant des technologies modernes et en respectant les standards de l'entreprise. 
\\ \\
Cette solution doit permettre de gagner du temps, d'éviter les erreurs et de faciliter la communication entre les différents professionnels.

Il m'a été demandé de tester plusieurs outils et services (Relume, Weblow et Devlink) pour voir si ils peuvent répondre se répondres aux problématiques.

Ensuite, il m'a été demandé d'étudier les différentes possiblité d'interactions entre les outils. 

Puis, de créer un prototype de la solution en utilisant les technologies React et Next.js connecté a un Headless CMS.

Enfin, de tester la solution, de identifié les forces et les faiblesses pour présenter la solution à l'équipe.


