\chapter*{Synthèse}

Ce stage s'inscrit au sein de l'entreprise DEPT, au sein du département irlandais. DEPT est une entreprise pionnière en technologie et en marketing qui aide ses clients à garder une longueur d'avance. C'est une agence numérique à service complet, avec une équipe de plus de 4 000 spécialistes du numérique répartis sur plus de 30 sites sur cinq continents. Elle travaille à l'échelle mondiale avec des clients renommés tels qu'Adidas, Patagonia, Google Workspace et bien d'autres. DEPT combine narration créative et technologie pour créer des expériences numériques complètes, optimisant tout le parcours client. De plus, elle conçoit, construit et intègre des logiciels et du matériel pour les principales entreprises SaaS et traditionnelles.

Au cours de ce stage, ma mission a été de tester différents services, mettre en relation des outils, développer des solutions informatiques pour créer une UI et une UX totalement dynamiques. Afin de résoudre plusieurs problèmes comme la répétition du travail, le manque de synergie entre différents métiers tels que les designers graphiques, les content managers et les développeurs.

La solution développée pendant le stage permet de connecter la création de wireframes, le style graphique, le code et le contenu des pages en quelques clics pour permettre une meilleure expérience de travail et de relation entre les différents processus de création, afin de gagner en efficacité.

Ce stage m'a permis de découvrir le travail dans une entreprise internationale, l'ingénierie, la mise en place de projets, notamment en termes de gestion de projet, de processus d'automatisation et de différentes méthodes de travail. J'ai également eu l'opportunité de découvrir différents outils et services à la pointe de la technologie. J'ai également pu renforcer mes compétences en développement web en testant et en développant avec JavaScript, React et Next.js. Enfin, j'ai appris de nouvelles façons de coder avec l'apprentissage de nouveaux motifs d'architecture logicielle, de nouveaux outils et une nouvelle approche de l'utilisation de l'intelligence artificielle.

Grâce à ce stage, j'ai pu participer à un projet concret qui m'a permis de découvrir la mise en place et la réflexion de projets dans une entreprise internationale.
\\ \\
\textbf{Mots-clés :} Processus d'automatisation, Développement web, JavaScript, React, Next.js, Headless CMS, Service d'intelligence artificielle, Méthodes de travail, Motifs d'architecture logicielle, Gestion de projet.
