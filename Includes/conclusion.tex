\chapter{Conclusion}
En conclusion, ce stage au sein du projet GOLIAT a été une expérience enrichissante et stimulante. J'ai réussi à réaliser toutes les fonctionnalités prioritaires définies dans le cahier des charges, démontrant ma capacité d'adaptation et d'intégration au sein du projet. J'ai conçu et développé toutes les pages de l'interface en respectant la charte graphique établie, en proposant un design moderne et soigné, avec une attention particulière portée à l'expérience utilisateur et à la compatibilité responsive sur différentes plateformes. Les composants essentiels tels que la barre de navigation et le tableau triable et responsive ont été implémentés avec succès.

La mise en place de toutes les fonctionnalités de l'API a été un élément clé de ma contribution au projet. J'ai assuré la connexion à l'API, la lecture des caméras, l'ajout des caméras, la gestion des opérations, y compris la suppression filtrée par leur statut, ainsi que le traitement des photos des opérations avec le téléversement et l'obtention des points chauds, qui sont ensuite visualisés sur une carte interactive. Toutes ces fonctionnalités sont opérationnelles et parfaitement intégrées à l'API existante, grâce à l'utilisation de Docker pour connecter et exécuter l'ensemble du projet avec une seule ligne de commande.

Ce stage m'a permis d'acquérir de nouvelles compétences et de renforcer mes connaissances dans divers domaines. J'ai pu approfondir mes compétences en développement web, en utilisant des technologies telles que Next.js, React et TypeScript, ainsi qu'en gestion de bases de données avec PostgreSQL. J'ai également développé mes compétences en matière de gestion de projet, de résolution de problèmes et d'adaptation.
Cette expérience m'a apporté une compréhension approfondie du travail dans la recherche et l'ingénierie, ainsi que des méthodes de travail efficaces.

Je suis fier d'avoir contribué de manière significative au projet GOLIAT en créant une interface web fonctionnelle qui répond aux exigences du cahier des charges. Mon travail a été apprécié par l'équipe et ils m'ont proposé un contrat à durée déterminée de deux mois pour poursuivre et améliorer l'interface web. Cette opportunité de prolonger mon engagement témoigne de la confiance qu'ils ont en mes compétences et de l'importance accordée à mon apport au projet.

En ce qui concerne mes perspectives d'avenir, je nourris l'ambition de devenir développeur full stack spécialisé dans les médias interactifs. Je souhaite continuer mes études et me perfectionner dans les domaines du développement web, de la conception d'interfaces et des technologies émergentes. Ce stage a confirmé mon idée et ma vocation dans ce domaine passionnant. Je suis motivé à poursuivre mon parcours, à acquérir de nouvelles compétences et à contribuer à des projets innovants qui allient technologie et créativité.

En somme, ce stage m'a permis de grandir professionnellement et personnellement. J'ai pu mettre en pratique mes connaissances, relever des défis techniques et contribuer à un projet concret visant à améliorer la lutte contre les feux de forêts. Je suis reconnaissant d'avoir participé à cette expérience et je suis impatient de continuer à contribuer au développement de l'interface web dans le cadre du contrat qui m'a été proposé.