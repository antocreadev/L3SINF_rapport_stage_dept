\chapter{Conclusion}
Mon travail au sein de l'entreprise DEPT a été axé sur le développement d'une solution innovante visant à optimiser le processus de création d'interfaces utilisateur dynamiques. Ce projet a abouti à la création d'un prototype fonctionnel intégrant divers outils et technologies modernes.

Sur le plan de ma contribution à l'entreprise, j'ai réussi à créer une solution qui simplifie et accélère le processus de création, tout en améliorant la collaboration entre les différents métiers impliqués. Cette solution ouvre également la voie à de futures améliorations et évolutions, notamment en termes de tests et de déploiement à grande échelle. Les perspectives pour cette solution incluent son intégration dans les processus de développement de l'entreprise, ainsi que son éventuelle adoption par d'autres équipes ou projets.

Sur le plan personnel, ce stage m'a apporté une multitude d'enseignements et d'expériences enrichissantes. J'ai pu apprendre de nouvelles compétences techniques, notamment en matière de développement front-end et d'intégration avec des CMS headless. De plus, j'ai pu bénéficier d'un encadrement précieux de la part de M. Marc Raffalli, qui m'a guidé à travers les défis techniques et m'a permis de mieux comprendre l'architecture logicielle. Enfin, ce stage m'a offert l'opportunité de découvrir le fonctionnement d'une entreprise internationale, d'explorer de nouvelles méthodes de travail et de renforcer ma passion pour le développement web.

En somme, ce stage chez DEPT a été une expérience extrêmement enrichissante à la fois sur le plan professionnel et personnel. Il m'a permis de contribuer activement à un projet concret tout en développant mes compétences et en élargissant mes horizons dans le domaine du développement web.