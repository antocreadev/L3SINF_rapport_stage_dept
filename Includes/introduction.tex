\chapter{Introduction}

Ce rapport de stage retrace mon expérience au sein de l'entreprise DEPT, dans son département irlandais. DEPT est une entreprise innovante dans les domaines de la technologie et du marketing, permettant à ses clients de toujours rester en avance. En tant qu'agence numérique proposant une gamme complète de services, DEPT réunit plus de 4 000 experts en numérique, répartis sur plus de 30 sites à travers cinq continents. Collaborant à l'échelle mondiale avec des clients prestigieux comme Adidas, Patagonia, Google Workspace, et bien d'autres, DEPT marie la narration créative à la technologie pour offrir des expériences numériques globales, optimisant ainsi chaque étape du parcours client. En outre, l'entreprise conçoit, développe et intègre des logiciels et du matériel pour les principales sociétés SaaS et traditionnelles.

Dans le cadre de mon stage, ma mission consistait à tester divers services, interconnecter des outils et développer des solutions informatiques visant à créer une interface utilisateur (UI) et une expérience utilisateur (UX) entièrement dynamiques. L'objectif était de résoudre plusieurs problématiques, notamment la répétition des tâches et le manque de synergie entre différents métiers comme les designers graphiques, les content managers et les développeurs.

Le présent rapport se concentre plus particulièrement sur ma contribution au développement d'une solution innovante permettant de relier différents outils pour intégrer la création de wireframes, le style graphique, le code et le contenu des pages en seulement quelques clics. Cela visait à améliorer l'expérience de travail et la collaboration entre les différents processus créatifs, afin d'accroître l'efficacité.
\\ \\
La première partie de ce rapport présente le contexte général du projet. J'y expose mes motivations personnelles qui m'ont conduit à m'engager dans ce stage, ainsi que les objectifs et les projets de l'entreprise DEPT. Je détaille également le cahier des charges, la gestion du temps pendant le stage et la gestion des projets au sein de l'entreprise.


La deuxième partie constitue le cœur de mon travail durant ce stage. J'y décris l'intégration des différents outils en suivant plusieurs étapes. Tout d'abord, je présente ces outils en détail, en soulignant leurs fonctionnalités et leur utilité. Ensuite, j'expose les réflexions sur les diverses possibilités d'interactions entre ces outils, en expliquant les choix technologiques et les stratégies adoptées.

La troisième partie se concentre sur la logique de développement avec un Headless CMS, développée en collaboration avec l'équipe du projet. J'y détaille le processus de réflexion, les décisions prises et les raisons derrière ces choix.


Enfin, je termine par une analyse critique de la solution développée, en présentant ses avantages et ses inconvénients. Cette évaluation permet de mettre en lumière les points forts du projet tout en identifiant les axes d'amélioration possibles pour l'avenir.
\\ \\
Je conclurai ce rapport en mettant en évidence les compétences que j'ai développées tout au long de mon stage, notamment en matière de développement web, de gestion de projet et de résolution de problèmes. Je soulignerai l'importance du travail d'équipe etl'exploration de nouvelles technologie. Ce rapport permettra ainsi de mieux appréhender les différentes étapes de mon travail, les solutions techniques que nous avons apportées