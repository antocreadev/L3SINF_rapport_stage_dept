\chapter{Contexte et cahier des charges}
\section{Présentation des outils à conecter}

\subsection{Relume}
Relume est un outil innovant en SaaS, conçu pour révolutionner la création de sites web grâce à l'intelligence artificielle. Il permet de générer rapidement des plans de site et des wireframes UX, tout en s'intégrant de manière fluide avec Figma et Webflow via un simple processus de copier-coller. L’interface intuitive de Relume embarque les utilisateurs dès l’onboarding, où la description succincte d’un site suffit pour que l’IA crée automatiquement une arborescence complète. Ce processus non seulement économise du temps, mais rend également accessible la création de sites web professionnels.

De plus, Relume est particulièrement efficace pour les sites classiques, comme les sites e-commerce, qui utilisent souvent des composants récurrents tels que des carrousels, des barres de navigation, des pieds de page, etc. La génération automatique de ces éléments permet de gagner un temps précieux tout en garantissant une structure cohérente et professionnelle. Relume offre également la possibilité de générer des contenus textuels en français, bien que ceux-ci soient souvent génériques.

Idéal pour les projets simples et les entreprises sans expertise numérique interne, Relume se distingue par sa capacité à augmenter la productivité tout en simplifiant la gestion des projets web. Toutefois, pour des projets plus complexes nécessitant des designs et des contenus plus personnalisés, l’intervention humaine reste indispensable. 